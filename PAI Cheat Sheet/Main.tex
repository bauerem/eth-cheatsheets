\documentclass[11pt,landscape,a4paper,fleqn]{article}
\usepackage[utf8]{inputenc}
\usepackage[ngerman]{babel}
\usepackage{tikz}
\usepackage{mathtools}
\usepackage{bbm}
\usetikzlibrary{shapes,positioning,arrows,fit,calc,graphs,graphs.standard}
\usepackage[nosf]{kpfonts}
\usepackage[t1]{sourcesanspro}
\usepackage{scalerel}
\usepackage[vlined]{algorithm2e}
%\usepackage[lf]{MyriadPro}
%\usepackage[lf,minionint]{MinionPro}
\usepackage{multicol}
\usepackage{xcolor}
\usepackage{wrapfig}
\usepackage[top=4mm,bottom=4mm,left=4mm,right=3mm]{geometry} %used to be top=3mm,bottom=4mm,left=4mm,right=3mm
\usepackage[framemethod=tikz]{mdframed}
\usepackage{microtype}
\usepackage{paralist} % for compacter lists
\usepackage{bm}
\usepackage{algpseudocode}

\makeatletter
\def\BState{\State\hskip-\ALG@thistlm}
\makeatother


\let\bar\overline

\definecolor{myblue}{rgb}{0,.4,.8}
\definecolor{mygreen}{rgb}{0,.7,.6}
\definecolor{myorange}{cmyk}{0, .73, .73, .03}

\definecolor{darkgreen}{cmyk}{0.97,0,1,0.57}
\definecolor{mypink}{cmyk}{0, 0.7808, 0.4429, 0.1412}


\pgfdeclarelayer{background}
\pgfsetlayers{background,main}

\everymath\expandafter{\the\everymath \color{myblue}}
\everydisplay\expandafter{\the\everydisplay \color{myblue}}
\color{myblue}

\renewcommand{\baselinestretch}{.8}
\pagestyle{empty}

\global\mdfdefinestyle{header}{%
linecolor=gray,linewidth=1pt,%
leftmargin=0mm,rightmargin=0mm,skipbelow=0mm,skipabove=0mm,
}

\makeatletter
\renewcommand{\section}{\@startsection{section}{1}{0mm}%
                                {.2ex}%
                                {.2ex}%x
	                                {\color{myorange}\sffamily\small\bfseries}}
\renewcommand{\subsection}{\@startsection{subsection}{1}{0mm}%
                                {.2ex}%
                                {.2ex}%x
                                {\color{mygreen}\sffamily\bfseries}}
\renewcommand{\subsubsection}{\@startsection{subsubsection}{1}{0mm}%
	{.2ex}%
	{.2ex}%x
	{\sffamily\bfseries}}


% math helpers
\DeclareMathOperator*{\argmin}{argmin}
\DeclareMathOperator*{\argmax}{argmax}
\newcommand{\E}{\mathbb{E}}

\makeatother
\setlength{\parindent}{0pt}

\newcommand{\imp}[1]{\boxed{\boldsymbol{#1}}} % Einrahmung und Fett
\newcommand{\w}{\omega}
\newcommand{\ud}{\,\mathrm{d}}% Differential
\newcommand{\norm}[1]{\left\lVert#1\right\rVert}
\newcommand{\term}[1]{\textbf{#1}}
\newcommand{\X}{\mathcal{X}}

% compress equations
%\medmuskip=0mu
%\thinmuskip=0mu
%\thickmuskip=0mu
\begin{document}
\small
\begin{multicols*}{4}
    \section*{Basics}

Ideas for stuff to add: SSLN + WLLN + CLT (simple formulation on slide 2 from Extreme Value Theory slides)

HOPITAL'S RULE

INTEGRALS (IE HIGHER MOMENTS)

$\exp(x)=\lim_{n\rightarrow \infty} (1+\frac{x}{n})^n=\sum^\infty_{k=0}\frac{x^k}{k!}$



\section*{Probability}
\subsection*{Conditional Probability}
$\bbP(A|B) = \frac{\bbP(A\cap B)}{\bbP(B)}$

% Cummulative Distribution Function
\subsection*{Cummulative Distribution Function}
\[
    F_X(x) = \mathbb{P}[X \leq x] = \int_{-\infty}^x f(x') dx'
\]
Let $X$ be a random variable. The CDF $F_X : \mathbb{R} \to [0, 1]$ given by
$\mathbf{F_X(x) = \mathbb{P}[X \leq x]}$ satisfies
\begin{itemize}
\item Normal: $\lim_{x \to -\infty} F_X(x) = 0$ and
    $\lim_{x \to \infty}F_X(x) = 1$
\item Right-Continuity: $F_X(x_n) \downarrow F_X(x) \text{ for } x_n
  \downarrow x \in \mathbb{R}$
\item Monotonicity: $F_X(a) \leq F_L(b) \text{ for } a \leq b$
\end{itemize}

\subsection*{Multivariate}
$f_{Y|X=x}(y)=\frac{f_{X,Y}(x,y)}{f_X(x)}$

\subsection*{Expectation Value}
$\mathbb{E}[X] = \sum_{i=1}^\infty x_i p_i$

$\mathbb{E}[X] = \int_{-\infty}^\infty x \textcolor{red}{f(x)} dx$

$\mathbb{E}$ is linear: $\mathbb{E}[\sum_{i=1}^N a_i X_i]
= \sum_{i=1}^N a_i \mathbb{E}[X_i]$

\subsection*{Conditionality}

$\E[Y|X=x]=\int y f_{Y|X=x}(y) dy=\int y f_{X,Y}(x,y)/f_X(x) dy$

$\E[X]=\E[X|A]\mathbb{P}(A)+\E[X|A^c]\mathbb{P}(A^c)$

$\implies \E[X|A]=\E[1_A X|A]=\frac{\E[X1_A]}{\mathbb{P}(A)}$

\[
  \mathbb{E}[X|X \textcolor{red}{\geq} \mu] = \frac{1}{\textcolor{red}
  {\mathbb{P}}[X \geq \mu]} \int_\mu^\infty xf(x)dx
\]
with $\mathbb{P}[X\geq\mu] = \int_\mu^\infty f(x)dx$

\subsection*{Variance}
$Var(X) = \mathbb{E}[(X - \mathbb{E}[X])^2]
        = \mathbb{E}[X^2] - \mathbb{E}[X]^2$

\subsection*{Marginal PDF}
Given two continuous RV $X$ and $Y$ whose joint distribution is known, then:
\[
    f_X(x) = \int_c^d f_{X,Y}(x,y) dy \quad \text{resp.} \quad
    f_Y(y) = \int_a^b f_{X,Y}(x,y) dx
\]
for $x\in [a,b]$ and $y \in [c,d]$.

\subsection*{Marginal CDF}
If joint CDF is known:

\textbf{Discrete RV:}
\[
    F_{X,Y}(x,y) = \bbP[X \leq x, Y \leq y]
\]
\textbf{Continous RV:}
\[
    F_{X,Y}(x,y) = \int_a^x \int_c^y f_{X,Y}(x',y')dy'dx'
\]

If $X$ and $Y$ jointly take values on $[a,b] \times [c,d]$ then
\[
    F_X(x) = F_{X,Y}(x,d) \quad \text{and} \quad F_Y(y) = F_{X,Y}(b_y)
\]
If $\textcolor{red}{d = \infty}$ then
\[
    F_X(x) = \lim_{y \to \infty} F_{X,Y}(x,y)
\]
Likewise for $F_Y(y)$.

\subsection*{Empirical Distribution Function}
\textbf{Def:} Let $x_1, \dots, x_n$ be independent realizations of a random variable $X$. The
corresponding \textit{empirical distribution function} 
$\hat{F}_X : \mathbb{R} \to [0, 1]$ is given by the step function
\[
  \hat{F}_X(x) = \frac{1}{n} \sum_{i=1}^n 1_{\{x_i \leq x\}} \qquad x\in
  \mathbb{R}.
\]

\textbf{Law of Large Numbers (LLN) for the edf:}

For all $x\in\R: \hat{F}_n(x)\rightarrow F(x)$ as $n\rightarrow \infty\; \mathbb{P}-$a.s.

\textbf{Glivenko-Cantelli Theorem for the edf:}

$\sup_{x\in\R} | \hat{F}_n(x) - F(x)|\rightarrow 0$ as $n\rightarrow \infty\; \mathbb{P}-$a.s.

\textbf{Statistical tests for the edf:}
\begin{itemize}
    \item \textbf{Kolmogorov-Smirnov:} \\
    $T_n=\sup_{x\in\R} |\hat F_n(x) - F(x)|$
    \item \textbf{Cramér-von Mises:} \\
    $T_n=n\int_\R [\hat F_n(x) - F(x)]^2 dF(x)$
    \item \textbf{Anderson-Darling:} \\
    $T_n=n\int_\R \frac{[\hat F_n(x) - F(x)]^2}{F(x)(1-F(x))} dF(x)$
    \item \textbf{Jarque-Bera (if $F\sim \mathcal{N}(\mu, \sigma^2)$):} \\
    Let $\hat \beta_n$ and $\hat \kappa_n$ be sample versions of
    \[\text{skewness: } \beta=\frac{\E[(X-\mu)^3]}{\sigma^3}\; \text{ \&}\]
    \[\text{kurtosis } \kappa=\frac{\E[(X-\mu)^4]}{\sigma^4}.\]
    Then under the null-hypothesis, for large $n$:
    \[\frac{n}{6}\big (\hat \beta_n ^ 2 + \frac{(\hat \kappa_n-3)^2}{4} \big)\sim\chi^2_2\]
\end{itemize}

\textbf{Graphical tests for the edf:} \\
Denote $x_{(1)}\leq\cdots\leq x_{(n)}$ ordered sample and \\
$\hat{F}_X(x) = \frac{1}{n} \sum_{i=1}^n 1_{\{x_i \leq x\}}=\frac{1}{n} \sum_{i=1}^n 1_{\{x_{(i)} \leq x\}}$
\begin{itemize}
    \item \textbf{P-P Plot:} \\
        \[\text{Graph} = \big\{ (p_i, F(x_{(i)})): i\in [n]\text{ and }\]
        \[p_i=\frac{i-1/2}{n}\approx \frac{i}{n}\approx\hat F_n(x_{(i)})\}\]
        If points are close to diagonal then $\hat F_n \approx F$.
    \item \textbf{Q-Q Plot:} \\
        \[\text{Graph}=\{(q(p_i), F(x_{(i)})): i\in [n] \big\} \]
        where $u\mapsto q(u)$ is a quantile function of $F$. \\
        If points are close to diagonal then $\hat F_n \approx F$.
    \begin{itemize}
    \item tail differences better visible than in P-P
    \item
        If $\hat F_n \approx F=\mathcal{N}(\mu, \sigma^2)$ then Graph approximately follows: $y=\mu+\sigma x$
    \item S-shaped $\implies$ heavier-tailed than $\mathcal{N}$
    \item Daily returns typically have kurtosis $\kappa > 3$. They are ``leptokurtic'': narrower center, heavier tails than $\mathcal{N}(\mu,\sigma^2)$ for which $\kappa=3$
    \end{itemize}
\end{itemize}


\subsection*{Bayes Theorem}
$P(A | B) = \frac{P(B | A)P(A)}{P(B)}$
\subsection*{Law of Total Probability}
$P(A) = \sum_n P(A | B_n) P(B_n)$ where $B_n$ discrete and finite.
\paragraph{Example:} $Y \sim Bernoulli$ and $Z \sim Exp(\lambda)$. \\
$\mathbb{P}[YZ \leq x] = \mathbb{P}[YZ \leq x | Y = 1]\mathbb{P}[Y=1]
+ \mathbb{P}[YZ \leq x | Y = 0]\mathbb{P}[Y = 0]$

\subsection*{Student-t}
PDF:
\[
    f_\nu(x) = c \big(1+\frac{x^2}{\nu}\bigg)^{-\lambda}
\]
for
\[
    c = \frac{\Gamma(\frac{\nu+1}{2})}{\sqrt{\nu\pi}} \Gamma(\frac{\nu}{2})
    \qquad
    \lambda = \frac{\nu+1}{2}
\]
where for any positive integer: $\Gamma(n) = (n-1)!$



TODO: amma function / S3A4

\sep



\section*{Common Distributions}
TODO: Add chi squared

% Normal distribution
\subsection*{Normal}
\textbullet Can NOT use Gaussian Integral for Gaussian Functions!
\textbullet Use a sub for computations!

TODO: See S2A1. Note: See the question I asked in the mathematics Discord.
TODO: Add Normaldistribution

\[
    f(x) = \frac{1}{\sqrt{2\pi\sigma^2}}e^{-\frac{(x-\mu)^2}{2\sigma^2}}
    , \quad \mu\in\R, \sigma^2 > 0
\]
$\E[X] = \mu$ (See example) \\
$\text{Var}[X] = \sigma$ (See example)

\subsection*{Bernoulli}
\warning Takes value $1$ with prob. $p$ and $0$ with prob. $q = 1 - p$.
\warning Binomial Dist. with $n=1$.

\[
  f(k;p) = \begin{cases}
    p & k = 1 \\
    q = 1-p & k = 0
  \end{cases}
\]
$f(k;p) = p^k(1-p)^{1-k} = pk + (1-p)(1-k)$ for $k \in (0, 1)$ 

$\mathbb{E}[X] = Pr(X = 1) \cdot 1 + Pr(X = 0)\cdot 0 = p$ \\
$\mathbb{E}[X^2] = p = \mathbb{E}[X]$ \\
$Var[X] = pq = p(1-p)$


% Poisson
\subsection*{Poisson: Pois($\lambda$)}
\[
  \mathbb{P}[N = n] = e^{-\lambda}\frac{\lambda^n}{n!}
  \quad n = \textcolor{red}{0}, 1, 2, \dots
\]

$\mathbb{E}[N] = \lambda$ (Use Series for exp.)

$\mathbb{E}[N^{\textcolor{red}{2}}] = \lambda^2 + \mathbb{E}[N] = \lambda^2 +  \lambda$ (Use factorial trick)

$Var[N] = \lambda = \mathbb{E}[N]$

% Exponential
\subsection*{Exponential: $\text{Exp}(\lambda)$ w. $\lambda>0$}

PDF: $f(x) = \lambda e^{-\lambda x} 1\{ x \geq 0 \}$; CDF: $F(x) = (1- e^{-\lambda x}) 1\{ x \geq 0 \}$

$\E[X] = \frac{1}{\lambda}$ \& $E[X^{\textcolor{red}{2}}] = \frac{2}{\lambda^2}$ $\implies$ $Var[X] = \frac{1}{\lambda^2}$


\subsection*{Uniform}
$\Var U=\frac{(b-a)^2}{12}$; Quantile: $q(u)=-\log(1-u)/\lambda$


\subsection*{Gamma distribution}
$f_{\Gamma(\alpha,\beta)}(x)=\beta^\alpha x^{\alpha-1}e^{-\beta x}/\Gamma (\alpha)$

\subsection*{Pareto}
TODO: Add pareto distribution. See S1P4.c

\sep

\section*{Mathematical Tools}
\subsection*{Integrals}
\begin{align*}
    \int 1\cdot \log(x) dx &= x(\log(x)-1) \\
    \int_{-\infty}^\infty e^{-x^2} dx &= \sqrt{\pi} \\
    \int_{-\infty}^\infty e^{-x^2/2} dx &= \sqrt{2\pi}
\end{align*}

\subsection*{Partial Integration}
\[
  \int_a^b u v' dx = [u v]^b_a - \int_a^b u' v dx
\]

\subsection*{Direct Integration}
\[
    \int f(g(x))g'(x)dx = F(g(x))
\]

\subsection*{Identities}
\textbf{Factorials:} $\frac{n^2}{n!} = \frac{n(n-1) + n}{n!} = \frac{1}{(n-2)!} + \frac{1}{(n-1)!}$

\section*{Concepts}
\subsection*{Main Goal of Regulation}
Ensure that financial institutions have enough capital to remain solvent.

\subsection*{Three Pillars of Financial Regulation}
The three pillar concept is at the basis of the Basel II and Solvency II
regulatory frameworks.

\paragraph{Pillar 1: Minimal Capital Charge} Requirement for the calculation of
the regulatory capital to ensure that a bank holds sufficient capital for its
\textcolor{green}{credit risk} in the \textit{banking book},
\textcolor{green}{market risk} in the \textit{trading book} and
\textcolor{green}{operational risk}.

\paragraph{Pillar 2: Supervisory Review Process} Local regulators review the
checks and balances put in place for capital adequacy assessments, ensure that
banks have adequate regulatory capital and perform stress tests of a bank's
capital adequacy.

\paragraph{Pillar 3: Market Discipline}
Banks are required to make their risk management processes more transparent.

\subsection*{Model Uncertainty}
Model uncertainty refers to the uncertainty about the accuracy of a model. It
results from imprecise and idealized assumptions, which, to some degree, have
to be made in every modeling framework.

\subsection*{Knightian Uncertainty}
In economics, Knightian uncertainty is a lack of any quantifiable knowledge 
about some possible occurrence, as opposed to the presence of quantifiable
risk. The concept acknowledges some fundamental degree of ignorance, a limit 
to knowledge, and an essential unpredictability of future events.

\subsection*{Ambiguity Aversion}
In decision theory and economics, ambiguity aversion is a preference for known 
risks over unknown risks. An ambiguity-averse individual would rather choose 
an alternative where the probability distribution of the outcomes is known 
over one where the probabilities are unknown.

TODO: Pic

\subsection*{Innovation}
In time series analysis (or forecasting) — as conducted in statistics, signal processing, and many other fields — the innovation is the difference between the observed value of a variable at time t and the optimal forecast of that value based on information available prior to time t. If the forecasting method is working correctly, successive innovations are uncorrelated with each other, i.e., constitute a white noise time series.
\subsection*{Empirical Distribution Function}
Let $x_1, \dots, x_n$ be independent realizations of a random variable $X$. The
corresponding \textit{empirical distribution function} 
$\hat{F}_X : \mathbb{R} \to [0, 1]$ is given by the step function
\[
  \hat{F}_X(x) = \frac{1}{n} \sum_{i=1}^n 1_{\{x_i \leq x\}} \qquad x\in
  \mathbb{R}
\]
whereas $1_{\{\cdot\}}$ is the indicator function.

\textbf{An application: Generating Loss Distr.}
We can approximate the dist. of $L_{t+1}$ by the edf above:
\[
  \hat{F}_X(x) = \frac{1}{n} \sum_{i=1}^n 1_{\{\textcolor{red}{l_{l-i+1}}
  \leq x\}}
\]
where $l_{t-n+1}, \dots, l_t$ are the last $n$ realized losses. Each of those
losses has equal probability (hence the $1/n$ and the indicator function).

\textbf{Advantage} Easy to implement, no modeling assumptions, no eastimation
required (you just look at past losses).

\textbf{Drawbacks} Sufficient data for all risk-factors required, makes
predictions based on past data.




%\sep

%On slide 12 of the extreme value slides there is a statement involving \(H^{\theta}\).

%If I'm not mistaken only \(H_{\xi}\), \(H_{\xi,\mu, \sigma}\) are defined.

%So what is \(H^{\theta}\)? Is it \(H_{1/\theta}\) or \(H_{\theta}\) or something else?

%Similarly, \(H^{\theta}_{\xi}\) is mentioned below.

%Since \(\theta\) has a value restricted to (0,1], I suppose that the \(H^{\theta}\) is parameterized by \(\theta\) and the significance of \(\theta\) is not merely to distinguish it from \(H\)
	\section{Bayesian Linear Regression}
\term{RR predict.:}
$P(y^*|x_{1:n},y_{1:n},x^*)=\mathcal{N}(\bar{\mu}^Tx^*,\sigma_n^2)$
\subsection*{MAP estimation=Ridge Regression}
If $P(w)=\mathcal{N}(0,\sigma_p^2I)$ and $P(y_{1:n}|w,x_{1:n})=\prod_{y_i} \mathcal{N}(w^Tx_i,\sigma_n^2)$, then $\argmax P(w|x_{1:n},y_{1:n})=\argmin \sum_i (y_i-w^Tx_i)^2+\frac{\sigma_n^2}{\sigma_p^2}||w||^2$.
\subsection*{Bayesian Linear Regression}
\term{Full posterior distr.:} $P(w|X,y)=\mathcal{N}(\bar{\mu},\bar{\Sigma})$, w.

$\bar{\mu}=(X^TX+\sigma_n^2I)^{-1}X^Ty$, 
$\bar{\Sigma}=(\sigma_n^{-2}X^TX+I)^{-1}$

\term{Predictions:} If $w\sim\mathcal{N}(\bar{\mu},\bar{\Sigma})$, $f^*=w^Tx^*$ and $y^*=f^*+\epsilon\sim\mathcal{N}(f^*,\sigma_n^2)$, then
\begin{inparaitem}[$\color{mygreen} \triangleright$]
\item $P(f^*|x_{1:n},y_{1:n},x^*)=\mathcal{N}(\bar{\mu}^Tx^*,x^{*T}\bar{\Sigma}x^*)$,
\item $P(y^*|x_{1:n},y_{1:n},x^*)=\mathcal{N}(\bar{\mu}^Tx^*,x^{*T}\bar{\Sigma}x^*+\sigma_n^2)$ $=\int p(y^*|x^*,w)p(w|x_{1:n},y_{1:n})dw$.
\end{inparaitem}

\subsection*{Aleatoric vs. epostemic uncertainty}
\term{Aleatoric} uncertainty about $y^*|f^*$: $\sigma_n^2$

Generally: Irreducible noise

\term{Epistemic} uncertainty about $f^*$: $x^{*T}\bar{\Sigma}x^*$ 

Generally: Model uncertainty due to lack of data

\subsection*{Aleatoric vs. epostemic uncertainty}
Need to choose $\sigma_n^2$ and $\sigma_p^2$.

\term{Algorithm:}
\begin{inparaitem}[$\color{mygreen} \triangleright$]
\item Choose $\hat{\lambda}=\frac{\hat{\sigma}_n^2}{\hat{\sigma}_p^2}$ via CV

\item Estimate $\hat{\sigma}_n^2=\frac{1}{n}\sum(y_i-\hat{w}^T x_i)^2$

\item Solve $\hat{\sigma}_p^2=\frac{\hat{\sigma}_n^2}{\hat{\lambda}}$
\end{inparaitem}
	\section{Gaussian Processes}

	\section{Variational Inference}
	\section{Markov Chain Monte Carlo}
	\section{Bayesian Deep Learning}
	\section{Active Learning}
	\section{Markov Decision Processes}
	\section{Reinforcement Learning}

 Agent actions change state. State change $\sim$ unknown MDP.

- \textcolor{blue}{On}-policy: agent has full control (actions)

- \textcolor{blue}{Off}-policy: no control, only observational data 


\vspace*{1mm}
\subsection*{Model-free RL} {\fontsize{9.5}{6}\selectfont Directly estimate value function}

\textbf{TD-Learning:} {\fontsize{9}{6}\selectfont \textcolor{blue}{(On)}} Follow $\pi$, get $(x,a,r,x')$. 

Update: \mhl{$\hat{V}^\pi(x) \leftarrow (1 - \alpha_t) \hat{V}^\pi(x) + \alpha_t (r + \gamma \hat{V}^\pi(x'))$}

Thm: $\alpha_t \vDash RM$ and all $(x,a)$ pairs chosen $\infty$ often, then $\hat{V}$ converges to $V^\pi$ w.p. 1.


\textbf{\textcolor{darkgreen}{Optimistic} Q-learning} {\fontsize{9}{6}\selectfont \textcolor{blue}{(Off)}} Estimate $Q^*(x,a)$
%, $V^*(x), \pi^*(x)$ can be derived.

1) Init estimate / $\textcolor{darkgreen}{Q(x,a) = \frac{R_{max}}{1 - \gamma} \prod_{t=1}^{T_{init}} (1 - \alpha_t)^{-1}}$

2) Pick $a$ (e.g. $\epsilon_t$ greedy), get $(x,a,r,x')$, update: \mhl{$Q(x,a) \leftarrow (1 - \alpha_t) Q(x,a) + \alpha_t (r + \gamma \max_{a'} Q(x', a'))$}

Test time: greedy $\pi_G(x) = \arg\max_a Q(x,a)$

Thm: $\alpha_t \vDash RM$, all $(x,a)$ pairs chosen $\infty$ often, then $Q$ converges to $Q^*$ w.p. 1. \textcolor{violet}{Thm(*)} holds.

Computation time: $\mathcal{O}(|A|)$, Memory: $\mathcal{O}(|X||A|)$

%\textbf{SARSA} {\fontsize{9}{6}\selectfont \textcolor{blue}{(On)}}-policy version of Q-learning

\subsection*{RL via Function Approx} {\fontsize{9.5}{6}\selectfont Learn parametric approx. of (action) value function $V(x; \theta), Q(x,a;\theta)$}

\vspace*{1mm}
\textbf{TD-learning as SGD} {\fontsize{9}{6}\selectfont \textcolor{blue}{(On)}}: Tabular TD update rule can be viewed as SGD on loss $l_2(\theta; x, x', r) = \frac{1}{2}(V(x;\theta) - r - \gamma V(x'; \theta_{old})^2$. Then, $V \leftarrow V - \alpha_t \nabla_{V(x;\theta)} l_2$ is equiv. to TD update.

\textbf{Function Approx Q-learning} {\fontsize{9}{6}\selectfont \textcolor{blue}{(Off)}} \textcolor{red}{slow}

Loss $l_2(\theta;x,a,r,x') = \frac{1}{2}\delta^2$ where
\mhl{$\delta = Q(x,a;\theta) -$} \mhl{$r - \gamma \max_{a'}Q(x',a';\theta)$}. Alg: Until converged:

State $x$, pick action $a$, observe $r,x'$. Update: $\theta \leftarrow \theta - \alpha_t \nabla_\theta l_2$
$\Leftrightarrow$ \mhl{$\theta \leftarrow \theta - \alpha_t \delta \nabla_\theta Q(x,a;\theta)$}

\textbf{DQN} {\fontsize{9}{6}\selectfont \textcolor{blue}{(Off)}}: Q-learning with NN as func. approx. Use experience replay data $D$, cloned network to maintain constant NN across episode.

\mhl{$L(\theta) = \hspace*{-4mm} \sum\limits_{(x,a,r,x') \in D} \hspace*{-4mm} (r + \gamma \textcolor{red}{\max_{a'} Q(x', a'; \theta^{old})} - Q(x,a;\theta))^2$}

\textbf{Double DQN} {\fontsize{9}{6}\selectfont \textcolor{blue}{(Off)}}: Current NN to evaluate

action $\arg\max$; prevents maximization bias.

$L^{{\scaleobj{.55}{ DDQN}}} (\theta) = \hspace*{0mm} \sum_{(x,a,r,x') \in D} \hspace*{0mm} [r + \gamma \max_{a'} Q(x', a^*(\theta); \theta^{old})$

$ - Q(x,a;\theta) ]^2$,
$a^*(\theta) = \arg\max_{a'} Q(x', a'; \theta)$

\textcolor{red}{$\textcolor{red}{a_t = \arg\max_a Q(x_t,a;\theta)}$ intractable for $\textcolor{red}{|A|}$ large}


\subsection*{Policy Gradient Methods} Parametric policy $\pi_\theta$

Maximize $J(\theta) = \mathbb{E}_{\tau \sim \pi_\theta} [r(\tau)]$ ($\tau = x_{0:T}, y_{0:T}$), $r(\tau) = \sum_{t=0}^{T} \gamma^t r(x_t, a_t)$); via $\nabla_\theta$ {\fontsize{9}{6}\selectfont \textcolor{blue}{(On)}}. Theorem:

\mhl{$\nabla_\theta J(\theta) = \nabla_\theta \mathbb{E}_{\tau \sim \pi_\theta} r(\tau) = \mathbb{E}_{\tau \sim \pi_\theta} [r(\tau) \nabla_\theta \log \pi_\theta(\tau)]$}

MDP: \mhl{$\pi_\theta(\tau) = p(x_0) \prod_{t=0}^{T} \pi(a_t | x_t; \theta) p(x_{t+1} | x_t, a_t)$}

Thus: \mhl{$\nabla_\theta J(\theta) = \mathbb{E}_{\tau \sim \pi_\theta} [r(\tau) \sum_{t=0}^{T} \nabla_\theta \log \pi(a_t | x_t; \theta)]$}

Reducing variance via baselines:

\mbox{\fontsize{9.4}{6}\selectfont $\mathbb{E}_{\tau \sim \pi_\theta} [r(\tau) \nabla \log \pi_\theta(\tau)] = \mathbb{E}_{\tau \sim \pi_\theta} [\textcolor{red}{(r(\tau) - b)} \nabla \log \pi_\theta(\tau)]$}

\textbf{Rew2Go:} \mbox{$G_t = \sum_{t' = t}^{T} \gamma^{t' - t} r_{t'}$; $b_t(x_t) = \nicefrac{1}{T} \sum_{t=0}^{T-1} G_t$}

% Basic REINFORCE gradient estimate
\mhl{$\nabla J_T(\theta) = \mathbb{E}_{\tau \sim \pi_\theta} [\sum_{t=0}^T \gamma^t \textcolor{darkgreen}{G_t} \nabla_\theta \log \pi(a_t | x_t; \theta)]$}

Mean over returns: \textcolor{darkgreen}{replace $\textcolor{darkgreen}{G_t}$ with $(G_t - b_t(x_t))$}



\textbf{REINFORCE} \textcolor{blue}{(On)}: Input $\pi(a | x; \theta)$, init $\theta$

Repeat: generate episode $(x_i, a_i, r_i), i=0:T$;

for $t=0:T$: set $\textcolor{darkgreen}{G_t}$, update $\theta$:

\mhl{$\theta = \theta + \eta \gamma^t \textcolor{darkgreen}{G_t} \nabla_\theta \log \pi(A_t | X_t; \theta)$}




\textbf{Advantage Func:} {\fontsize{9.8}{6}\selectfont $A^\pi(x,a) = Q^\pi(x,a) - V^\pi(x)$}

$\forall x,a: A^{\textcolor{red}{\pi^*}}(x,a) \leq 0$; $\forall \pi,x: \max_a A^\pi(x,a) \geq 0$



\subsection*{Actor-Critic} \textcolor{blue}{(On)} Approx both $V^\pi$ \textit{and} policy $\pi_\theta$ (e.g. 2 NNs). Reinterpret score gradient:

\mhl{$\nabla J(\theta_\pi)$}$= \hspace*{-2mm} \underset{\tau \sim \pi_\theta}{\mathbb{E}} \hspace*{-1mm} [\sum_{t=0}^\infty \gamma^t Q(x_t, a_t; \theta_Q) \nabla \log \pi(a_t | x_t; \theta_\pi)]$

%$ = \mathbb{E}_{x \sim \rho^\theta, a \sim \pi_\theta(x)} [Q(x,a;\theta_Q) \nabla \log \pi(a | x; \theta_\pi)] = $

\mhl{$ =: \mathbb{E}_{(x,a) \sim \pi_\theta} [Q(x,a;\theta_Q) \nabla_{\theta_\pi} \log \pi(a | x; \theta_\pi)]$}

Allows online updates:

$\theta_\pi \leftarrow \theta_\pi + \eta_t \textcolor{darkgreen}{Q(x,a;\theta_Q)} \nabla \log \pi(a | x; \theta_\pi)$

$\theta_Q \leftarrow \theta_Q  - \eta_t \delta \nabla Q(x,a;\theta_Q)$ (FA Q-learning)

Variance redution: \textcolor{darkgreen}{replace with} $Q(x,a;\theta_Q) - V(x; \theta_V)$: advantage func. estimate $\rightarrow$ A2C

\subsection*{Off-policy Actor Critic \textcolor{blue}{\textnormal{(Off)}}}

Replace $\textcolor{red}{\max_{a'} Q(x', a'; \theta^{old})}$ in DQN $L(\theta)$ by $\textcolor{blue}{\pi(x'; \theta_\pi)}$, where $\pi$ should follow the greedy policy to model $\textcolor{red}{\max_{a'}}$. This is equivalent to:

\mhl{$\theta_\pi^* \in \arg\max_\theta \textcolor{violet}{\mathbb{E}_{x \sim \mu} [Q(x,\pi(x;\theta); \theta_Q)]}$},
where $\mu(x) > 0$ 'explores all states'. If $Q(\cdot; \theta_Q), \pi(\cdot; \theta_\pi)$ diff'able, use backprop to get stoch. gradients.

$\nabla_\theta \textcolor{violet}{J(\theta)} = \mathbb{E}_{x \sim \mu} [\nabla_\theta Q(x,\pi(x;\theta); \theta_Q)]$

$\nabla_{\theta} Q(x,\pi(x;\theta) = \nabla_a Q(x,a)|_{a = \pi(x;\theta)} \cdot \nabla_{\theta} \pi(x; \theta)$

Needs \textit{deterministic} $\pi$. Inject additional action noise (e.g. $\epsilon_t$ greedy) to ensure exploration.

{\fontsize{9.5}{6}\selectfont \textbf{Deep Deterministic Policy Gradient (DDPG)}}

1) init $\theta_Q, \theta_\pi$ 2) repeat: observe $x$, execute $a = \pi(x; \theta_\pi) + \epsilon$, observe $r,x'$, store in $D$. If time to update: for ITER: sample $B$ from $D$, compute targets
$y = r+ \gamma Q(x', \pi(x', \theta_\pi^{old}), \theta_Q^{old})$, update
\iffalse
do GD ($\theta_Q$)/ GA ($\theta_\pi$), update $\theta^{old} \leftarrow (1 - \rho) \theta^{old} + \rho \theta$
\fi

\iftrue
Critic: $\theta_Q \leftarrow \theta_Q - \eta \nabla \nicefrac{1}{|B|} \sum_B (Q(x,a;\theta_Q) - y)^2$,

Actor: $\theta_\pi  \leftarrow \theta_\pi + \eta \nabla \nicefrac{1}{|B|} \sum_B Q(x, \pi(x; \theta_\pi); \theta_Q)$,

Params: $\theta_j^{old} \leftarrow (1 - \rho) \theta_j^{old} + \rho \theta_j$ for $j \in \{\pi, Q \}$
\fi


\textbf{Randomized policy DDPG:} For Critic: sample $a' \sim \pi(x'; \theta_\pi^{old})$ to get unbiased $y$ estimates. For Actor: consider $\nabla_{\textcolor{red}{\theta_\pi}} \mathbb{E}_{a \sim \pi(x; \textcolor{red}{\theta_\pi})} Q(x,a;\theta_Q)$

Reparametrization trick: $a = \psi(x; \theta_\pi, \epsilon)$

$\nabla_{\theta_\pi} \mathbb{E}_{a \sim \pi_{\theta_\pi}} Q(x,a;\theta_Q) = \mathbb{E}_\epsilon \nabla_{\theta_\pi} Q(x, \psi(x; \theta_\pi, \epsilon); \theta_Q)$


	\section{Model Based (Deep) RL}

%\subsection*{Model-based (Deep) RL} {\fontsize{9.5}{6}\selectfont Learn MDP, optimize $\pi$ on it}

MLE estimate from path trajectory $\tau$:

{\fontsize{9.7}{6}\selectfont $P(X_{t+1} | X_t, A) \approx \frac{Cnt(X_{t+1}, X_t, A)}{Cnt(X_t, A)}$;
$r(x,a) \approx \nicefrac{1}{N_{x,a}} \hspace*{-5mm} \sum\limits_{t: X_t = x, A_t = a} \hspace*{-5mm} R_t$}

\textbf{$\mathbf{\epsilon_t}$ greedy:} Tradeoff exploration-exploitation
W.p. $\epsilon_t$: rand. action; w.p. $1 - \epsilon_t$: best action.
If $\epsilon_t \vDash RM$ $\implies$ converge to $\pi^*$ w.p. 1.

\textbf{Robbins Monro (RM):} $\sum_t \epsilon_t = \infty$, $\sum_t \epsilon_t^2 < \infty$

\textbf{$\mathbf{R_{max}}$ Algorithm:} Set unknown $r(x,a)$ to $R_{max}$, $r(x,a) \leq R_{max}, \forall x,a$, add \mhl{fairy tale state $x^*$}, set $P(x^* | x,a) = 1$, compute $\pi$. Repeat: run $\pi$ while updating $r(x,a)$, $P(x' | x,a)$, then recompute $\pi$.

\textcolor{violet}{Thm(*)}: W.p. $1 - \delta$, $R_{max}$ will reach $\epsilon$-opt policy in \#steps poly in $|X|, |A|, T, \nicefrac{1}{\epsilon}, \log(1 - \delta), R_{max}$.

Note: MDP is assumed ergodic.

\textbf{Problems of Model-based RL:} - Memory required: $P(x'|x,a) \approx \mathcal{O}(|X|^2 |A|)$, $r(x,a) \approx \mathcal{O}(|X||A|)$

- Computation: repeatedly solve MDP (VI, PI)


\subsection*{Planning \textcolor{blue}{\textnormal{(Off)}} \textcolor{myblue}{\textnormal{(cont. obsv. states)}}}

\textbf{MPC (known deterministic dynamics)}

Assume known model $x_{t+1} = f(x_t, a_t)$, plan over finite horizon $H$. At each step $t$, maximize:

\mhl{$J_H(a_{t:t+H-1}) := \sum_{\tau = t:t+H-1} \gamma^{\tau - t} r_\tau(x_\tau(a_{t:\tau-1}), a_\tau)$}

$x_\tau(a_{t:\tau-1}) = f(f(...(f(x_t, a_t), a_{t+1})..))$

then carry out $a_t$, then replan.

Optimize via gradient based methods (diff. $r, f$, cont. action) or via random shooting.

\vspace*{-2mm}
\textbf{Random shooting:} Pick rand. samples $a_{t:t+H-1}^{(i)}$

\vspace*{-1mm}
and pick sample $i^* = \arg\max_i J_H(a_{t:t+H-1}^{(i)})$

\textbf{MPC with Value estimate:} $J_H(a_{t:t+H-1}) :=$
\mhl{$\sum_{\tau = t:t+H-1} \gamma^{\tau - t} r_\tau(x_\tau(a_{t:\tau-1}), a_\tau) + \gamma^H V(x_{t+H})$}

$H=1$: $J_1(a_t) = Q(x_t, a_t)$; $\pi_G = \arg\max_a J_1(a)$

\textbf{MPC (known stochastic dynamics)}

{\fontsize{10}{6}\selectfont $\max\limits_{a_{t:t+H-1}} \underset{x_{t+1:t+H}}{\mathbb{E}} [ \sum\limits_{\tau = t:t+H-1} \hspace*{-4mm} \gamma^{\tau - t} r_\tau + \gamma^H V(x_{t+H}) | a_{t:t+H-1} ]$}

%Expectation via \hl{MC trajectory sampling}: $x_{t+1} = f(x_t, a_t, \epsilon_t)$, get unbiased estimates of $J_H$ and approx via sample average.

\textbf{Parametrized policy:} ($H = 0 \Leftrightarrow$ DDPG obj.)

$J_H(\theta) = \underset{x_0 \sim \mu}{\mathbb{E}} [ \sum\limits_{\tau = 0:H-1} \hspace*{-4mm} \gamma^{\tau} r_\tau + \gamma^H Q(x_{H}, \pi(x_H, \theta)) | \theta ]$

\textbf{MPC (unknown dynamics):} follow $\pi$, learn $f, r, Q$ off-policy from replay buf, replan $\pi$.

BUT: point estimates have poor performance, errors compound $\rightarrow$ use bayesian learning:

Model distribution over $f$ (BNN, GP) and use (approximate) inference (exact, VI, MCMC,..).

\textbf{Greedy exploitation for model-based RL:} \textcolor{mypink}{(*)}

1) $D=\{\}$, prior $P(f|\{\})$ 2) repeat: plan new $\pi$ to maximize \mhl{$\max_\pi \mathbb{E}_{f \sim P(\cdot | D)} J(\pi, f)$}, rollout $\pi$, add new data to $D$, update posterior $P(f | D)$

\textbf{PETS algorithm:} Ensemble of NNs predicting cond. Gaussian transition distr., use MPC.

\textbf{Thompson Sampling:} Like greedy\textcolor{mypink}{*} BUT in 2) sample model \mhl{$f \sim P(\cdot | D)$} and then \mhl{$max_\pi J(\pi, f)$}

Use epistemic noise to drive exploration.

\textbf{Optimistic exploration:} Like greedy\textcolor{mypink}{*} BUT in 2) \mhl{$\max_\pi \max_{f \in M(D)} J(\pi, f)$}; with $M(D)$ set of plausible models given $D$.


	
% -*- root: Main.tex -*-

%\subsection{Misc}
%\textbf{Lagrangian:} $f(x,y) s.t. g(x,y) = c$\\
%$
%\mathcal{L}(x, y, \gamma) = f(x,y) - \gamma ( g(x,y)-c)
%$\\
%\textbf{Parametric learning}: model is parametrized with a finite set of parameters, like linear regression, linear SVM, etc. \\
%\textbf{Nonparametric learning}: models grow in complexity with quantity of data: kernel SVM, k-NN, etc.\\
%\textbf{Empirical variance}: Look for dense and sparse regions. Regularize so that sparse regions are not contained (decr. variance). Measure by Variance CV of some classifiers.

% -*- root: Main.tex -*-
%\section{Ensemble Methods}
%Use combination of simple hypotheses (weak learners) to create one strong learner.
%
%strong learners: minimum error is below some $\delta < 0.5$
%
%weak learner: maximum error is below $0.5$
%\begin{equation}
%f(x) = \sum_{i=1}^{n} \beta_i h_i(x)
%\end{equation}
%\textbf{Boosting}: train on all data, but reweigh misclassified samples higher.
%
%\subsubsection*{Decision Trees}
%\textbf{Stumps}: partition linearly along 1 axis\\
%$h(x) = sign(a x_i - t)$\\
%\textbf{Decision Tree}: recursive tree of stumps, leaves have labels. To train, either label if leaf's data is pure enough, or split data based on score.
%
%
%\subsubsection*{Ada Boost}
%Effectively minimize exponential loss.\\
%$f^*(x) = \argmin_{f\in F} \sum_{i=1}^{n} \exp(-y_i f(x_i))$\\
%Train $m$ weak learners, greedily selecting each one
%\begin{equation*}
%(\beta_i, h_i) = \argmin_{\beta,h} \sum_{i=1}^{n} \exp(-y_i (f_{i-1} (x_j) + \beta h(x_j)))
%\end{equation*}
%\begin{compactdesc}
%	\item $c_b(x) \text { trained with } w_i$ \\
%	\item $\epsilon_b = \sum\limits_i^n \frac{w_i^b}{\sum\limits_i^n w_i^b} I_{c(x_i) \neq y_i} $\\
%	\item $\alpha_b = log \frac{1-\epsilon_b}{\epsilon_b} $\\
%	\item $w^{b+1}_i = w^b_i \cdot exp(\alpha_b I_{y_i \neq c_b(x_i)})$
%\end{compactdesc}
%
%Exponential loss function
%
%Additive logistic regression
%
%Bayesian approached (assumes posteriors)
%
%Newtonlike updates (Gradient Descent)
%
%If previous classifier bad, next has heigh weight



%\subsubsection*{Fischer's Linear Discriminant Analysis (LDA)}
%Idea: project high dimensional data on one axis.
%
%Complexity: $\mathcal{O}(d^2n$ with $d$ number of classifiers\\
%$c=2, p=0.5, \hat{\Sigma}_- = \hat{\Sigma}_+ = \hat{\Sigma} \\
%y = sign(w^\top x + w_0) \\
%w = \hat{\Sigma}^{-1}(\hat{\mu}_+ - \hat{\mu}_-) \\
%w_0 = \frac{1}{2}(\hat{\mu}_-^\top \Sigma^{-1} \hat{\mu}_- - \hat{\mu}_+^\top \Sigma^{-1} \hat{\mu}_+)
%$

% -*- root: Main.tex -*-
%\section{Unsupervised Learning}
%\subsection*{Parzen}
%$
%\hat{p}_n = \frac{1}{n} \sum\limits_{i=1}^n \frac{1}{V_n} \phi(\frac{x-x_i}{h_n})
%$
%where $\int \phi(x)dx = 1$
%\subsection*{K-NN}
%$
%\hat{p}_n = \frac{1}{V_k} \text{ volume with } k \text{ neighbours}
%$
%\subsection*{K-means}
%$
%L(\mu) = \sum_{i=1}^{n} \min_{j\in\{1...k\}} \|x_i - \mu_y \|_2^2
%$\\
%
%\textbf{Lloyd's Heuristic}:\\ (1) assign each $x_i$ to closest cluster \\
%(2) recalculate means of clusters.
%
%Iteration over (repeated till stable):
%\begin{compactdesc}
%	\item[Step 1:]$ \text{argmin}_c ||x-\mu_c||^2$ \\
%	\item[Step 2:]$ \mu_\alpha = \frac{1}{n_\alpha} \sum \vec{x}$
%\end{compactdesc}

% -*- root: Main.tex -*-

%\section{Hidden-Markov model}
%State only depends on previous state.
%
%Always given: sequence of symbols $\vec{s} = \{s_1,s_2, \ldots s_n\}$
%\subsection*{Evaluation (Forward \& Backward)}
%Known: $a_{ij}, e_k(s_t)$
%
%Wanted: $P(X = x_i | S = s_t)$
%\begin{eqnarray}
%f_l (s_{t+1}) = e_l(s_{t+1}) \sum f_k(s_t) a_{kl} \\
%b_l(s_t) = e_l(s_t) \sum b_k(s_{t+1}) a_{lk} \\
%P(\vec{s}) = \sum_k f_k(s_n) a_k \cdot \text{ end} \\
%P(x_{l,t} | \vec{s}) = \frac{f_l(s_t) b_l(s_t)}{P(\vec{s})}
%\end{eqnarray}
%Complexity in time: $\mathcal{O}(|S|^2 \cdot T)$

%\subsection{Learning (Baum-Welch)}
%Known: only sequence and sequence space $\Theta$
%
%Wanted: $a_{ij}, e_k(s_t)$ \& most likely path $\vec{x} = \{x_1,x_2,\ldots x_n\}$
%
%\textbf{E-step I:} $f_k(s_t), b_k(s_t)$ by forward \& backward algorithm
%
%\textbf{E-step II:}
%\begin{eqnarray}
%P(X_t = x_k, X_{t+1} = x_l | \vec{s}, \Theta) = \\
%\frac{1}{P(\vec{s})} f_k(s_t) a_{kl} e_l(s_{t+1}) b_l(s_{t+1}) \\
%A_{kl} = \sum\limits_{j=1}^m \sum\limits_{t=1}^n P(X_t = x_k, X_{t+1} = x_l | \vec{s}, \Theta)
%\end{eqnarray}
%\textbf{M-step :}
%\begin{eqnarray}
%a_{kl} = \frac{A_{kl}}{\sum\limits_i^n A_{ki}} \text{   and   } e_k(b) = \frac{E_k(b)}{\sum_{b'} E_k(b')}
%\end{eqnarray}
%Complexity: $\mathcal{O}(|S|^2)$ in storage (space)
% -*- root: Main.tex -*-

%\subsection{Norms}
%\begin{inparadesc}
%	\item[\color{red}$l_0$:] $\|\mathbf{x}\|_0 := |\{i | x_i \neq 0\}|$
%	\item[\color{red}Nuclear:] $\|\mathbf{X}\|_\star = \sum_{i=1}^{\min(m, n)} \sigma_i$
	%			\item[\color{red}Euclidean:] $\|\mathbf{x}\|_2 := \sqrt{\sum_{i=1}^{N} \mathbf{x}_i^2} = \sqrt{\mathbf{x}^T \mathbf{x}} = \sqrt{\langle \mathbf{x}, \mathbf{x} \rangle}$
	%			\item[\color{red}$p$-norm:] $\|\mathbf{x}\|_p := \left( \sum_{i=1}^{N} |x_i|^p \right)^{\frac{1}{p}}$
	%\item[\color{red}Frobenius:] $\|\mathbf{A}\|_F :=\allowbreak %\sqrt{\sum_{i=1}^{M} \sum_{j=1}^{N} |\mathbf{A}_{i, j}|^2} =\allowbreak \sqrt{\operatorname{trace}(\mathbf{A}^T \mathbf{A})} =\allowbreak \sqrt{\sum_{i=1}^{\min\{m, n\}} \sigma_i^2}$ ($\sigma_i$ is the $i$-th singularvalue), $\mathbf{A} \in \mathbb{R}^{M \times N}$
%\end{inparadesc}
	
	%		\subsection*{Regularization}
%	The error term $L$ and the regularization $C$ with regularization parameter $\lambda$: $\min \limits_w L(w) + \lambda C(w)$\\
%	L1-regularization for number of features \\
%	L2-regularization for the length of $w$
	
%	\subsection*{Convex}
%	$\text{g(x) is convex}$\\
%	$\Leftrightarrow x_1,x_2 \in \mathbb{R}, \lambda \in [0,1]:$\\
%	$g(\lambda x_1) + (1-\lambda x_2) \leq \lambda g(x_1) + (1-\lambda) g(x_2)$
%	$ \Leftrightarrow g''(x) > 0$

%	\subsection*{Parametric to nonparametric linear regression}
%	Ansatz: $w=\sum_i \alpha_i x$\\
%	Parametric: $w^* = \underset{w}{\operatorname{argmin}} \sum_i (*Tx_i-y_i)^2 + \lambda ||w||_2^2$\\
%	$= \underset{\alpha_{1:n}}{\operatorname{argmin}} \sum \limits_{i=1}^n (\sum \limits_{j=1}^n \alpha_j x_j^T x_i - y_i)^2 + \lambda \sum \limits_i \sum \limits_j \alpha_i \alpha_j (x_i^T x_j)$\\
%	$= \underset{\alpha_{1:n}}{\operatorname{argmin}} \sum \limits_{i=1}^n (\alpha^T K_i - y_i)^2 + \lambda \alpha^T K \alpha$\\
%	$= \underset{\alpha}{\operatorname{argmin}} ||\alpha^T K -y||_2^2 + \lambda \alpha^T K \alpha$\\
%	Closed form: $\alpha^* = (K+\lambda I)^{-1} y$\\
%	Prediction: $y^*= w^{*^T} x = \sum \limits_{i=1}^n \alpha_i ^* k(x_i,x)$

\end{multicols*}
\end{document}